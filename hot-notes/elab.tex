\documentclass{article}
\usepackage{proof}
\usepackage{amssymb}
\usepackage{mathpartir}
\usepackage{stmaryrd}
\usepackage{MnSymbol}
\begin{document}

\newcommand{\elabMatch}[1]{#1 \textrm{ on } v : \tau_1 \textrm{ with } \vfail}
\newcommand{\elabPat}[1]{#1 \textrm{ on } e : \tau \textrm{ with } \vfail}
\newcommand{\ex}{\textrm{exp}}
\newcommand{\T}{\textrm{T}}
\newcommand{\sq}{\rightsquigarrow}
\newcommand{\E}{\mathcal{E}} 
\newcommand{\tr}{\triangleright}
\newcommand{\vfail}{v_\textrm{\small fail}}
\newcommand{\afail}{\alpha_\textrm{\small fail}}
\newcommand{\mfail}{m_\textrm{\small fail}}
\newcommand{\Min}[1]{\textrm{in}_{#1}}
\newcommand{\sandw}[1]{\triangleleft #1 \triangleright}
\newcommand{\unit}{{\times[]}}
\newcommand{\dec}{\textrm{dec}}
\newcommand{\headp}[1]{\llparenthesis\, #1\, \rrparenthesis}

\begin{center}
\LARGE { Elaboration Notes }
\end{center}

These notes were distilled and typeset by Jacob Potter \texttt{<jdpotter@andrew.cmu.edu>}, based on:
\begin{itemize}
\item Lectures and theory by Karl Crary \texttt{<crary@cs.cmu.edu>}
\item My own lecture notes and code
\item Lecture notes taken by Ben Segall \texttt{<bsegall@cs.cmu.edu>}
\end{itemize}

Some general formatting and implementation notes:

\begin{itemize}
\item ``$\Gamma \vdash \tau : \T$'' in a premise means that $\tau$ might have a free $\alpha$ in it, and so it must have inhabitant (fstsg $\sigma$) substituted for $\alpha$.
\item ``$\Gamma, \alpha:\sigma$'' means ``\texttt{extend G a (fstsg sg)}''.
\item ``$\Gamma \vdash \E \tr longid \textrm { in ...}$'' is \texttt{Resolve.resolve}.
\item ``$\Gamma \vdash \E \tr  ty\rightsquigarrow \tau$'' is \texttt{ElaborateType.elabTp}. 
\item ``$\Gamma \vdash \tau \equiv ... : \textrm{T}$'' means to pattern match on the output of \texttt{whnf} $\tau$.
\item The type that is called \texttt{unit} in ML is always written as $\unit$.
\end{itemize}

\section{elabMatch with $\vfail$}

\[ 
\inferrule[ {[ (pat, exp) ]} ]
{ \Gamma \vdash \E \tr \elabMatch{\textrm{pat}} \sq M : \sigma
\\ \Gamma, \alpha:\sigma \vdash \E @ (\alpha/m:\sigma) \tr \textrm{exp} \sq e : \tau_2
\\ \Gamma \vdash \tau_2 : \T
 }
{ \Gamma \vdash \E \tr \elabMatch{\textrm{pat} \Rightarrow \exp} \rightsquigarrow \textrm{let } \alpha/m = M \textrm{ in } (e:\tau_2) : \tau_2}
\]
\[
\inferrule[ (pat, exp) :: match ]
{ \Gamma \vdash \E \tr \elabMatch{\textrm{pat}} \sq M : \sigma
\\ \Gamma, \alpha:\sigma \vdash \E @ (\alpha/m:\sigma) \tr \textrm{exp} \sq e : \tau_2
\\ \Gamma \vdash \tau_2 : \T
\\  \Gamma \vdash \E \tr \elabMatch{\textrm{match}} \rightsquigarrow e' : \tau_2
 }
{
 \Gamma \vdash \E \tr \elabMatch{\textrm{pat} \Rightarrow \textrm{exp } | \textrm{ match}} \rightsquigarrow \\ \textrm{handle}(\textrm{let } \alpha/m = M \textrm{ in } (e:\tau_2), x, \textrm{iftag}(\vfail, x, _, e', \textrm{raise}_{\tau_2} \, x)) : \tau_2
 }
\]

\section{elabMatch}

\[
\inferrule
{
\Gamma \vdash \E \tr \elabMatch{\textrm{match}} \rightsquigarrow e : \tau
}{
\Gamma \vdash \E \tr \textrm{match} \textrm{ on } v : \tau_1 \rightsquigarrow (\textrm{let } \vfail = \textrm{newtag}_{\unit} \textrm{ in } e) : \tau
}
\]

\section{elabPattern}

\[
\inferrule[Pident]{ }
{
\Gamma \vdash \E \tr \elabPat{\textrm{id}} \sq (\Min{\textrm{VAL id}} \sandw{\textrm{id}}) : (\textrm{VAL id} : \sandw{\tau})
}
\]

\[
\inferrule[Pint]{ }
{
\Gamma \vdash \E \tr \elabPat{i} \sq \\
\left( \textrm{let } \_/\_ = \sandw{ \textrm{if } e = i \textrm{ then } \langle \rangle \textrm{ else raise}_\unit 
\left( \textrm{tag}_{\vfail} \langle \rangle \right)
} \textrm{ in } * :  1 \right) : 1
}
\]

Note: In the following rule, $\langle M1, M2 \rangle$ is shorthand for $\langle \_/\_ = M1, M2 \rangle$ and $\sigma_1 \times \sigma_2$ is shorthand for $\Sigma \_: \sigma_1 \times \sigma_2$.

\[
\inferrule[Ptuple]{
\Gamma \vdash \tau \equiv \times[\tau_0, ... \tau_{n-1} ] : \textrm{T} \\
\Gamma \vdash \E \tr pat_i \textrm{ on } \pi_i e : \tau_i \textrm{ with } \vfail \sq M_i : \sigma_i  \quad \left( \textrm{for $i \in [0, ..., n-1]$} \right)
} {
\Gamma \vdash \E \tr (pat_0, ... pat_n) \textrm{ on } e : \tau \textrm{ with } \vfail \sq \\
\langle M_0, \langle M_1, \langle ... \langle M_{n-1}, * \rangle \rangle \rangle 
: \sigma_0 \times (\sigma_1 \times (... (\sigma_{n-1} \times 1)))
}
\]


\[
\inferrule[Papp]
{
\Gamma \vdash \E \tr longid \textrm{ in VAL} \sq M_0 : (\textrm{DCON} : \sandw { \tau_0 }) \\
\Gamma \vdash \tau_0 \equiv \tau_c \times (\tau \rightarrow (\unit + \tau')) : \textrm{T} \\
\Gamma, \alpha : (\textrm{checkCon } \Gamma\  \tau') \vdash \E @ (\alpha/m:\sandw{\tau'}) \tr pat \textrm{ on Snd}(m) : \tau' \textrm{ with } \vfail \sq \sigma : M
}
{
\Gamma \vdash \E \tr longid\ pat \textrm{ on } e : \tau \textrm{ with } \vfail \sq \\
{\left(  \textrm{let } \alpha/m = \triangleleft \left( \begin{tabular}{@{}r@{}l@{}}
$ \textrm{case } (\pi_1 (\textrm{Snd}(\textrm{out } M_0)))\ e  \textrm{ of }$&$\textrm{inj}_0 \_
\Rightarrow \textrm{raise}_{\tau'} (\textrm{tag}(\vfail, \langle \rangle))$ \\
$|\ $ & $\textrm{inj}_1 x \Rightarrow x $
\end{tabular} \right) \triangleright \textrm{ in } M \right) : \sigma}}
\]

\section{elabTerm}

\texttt{Tint} and \texttt{Tstring} should be obvious. \texttt{Tprim} is handled with \texttt{ElaboratePrim.elabPrim}. Note that there are two rules for \texttt{Tvar} depending on the result of \textrm{resolve}.

\[\inferrule[Tvar : value]{
\Gamma \vdash \E \tr longid \textrm{ in VAL} \sq M : \sandw{\tau}}{
\Gamma \vdash \E \tr longid \sq \textrm{Snd}(M) : \tau }
\]

\[
\inferrule[Tvar : data constructor]{
\Gamma \vdash \E \tr longid \textrm{ in VAL} \sq M : (\textrm{DCON} : \sandw{\tau}) \\
\Gamma \vdash \tau \equiv \tau_c \times \tau_d : \textrm{T} }{
\Gamma \vdash \E \tr longid \sq \pi_0 (\textrm{Snd}(\textrm{out } M)) : \tau_c
} \]
%\[\infer{\Gamma\ent \epsilon\triangleright longid\rightsquigarrow \pi_0 (snd
  %(out\ M)) \colon\tau_C}


\[ \inferrule[Tapp]{
\Gamma \vdash \E \tr exp_1 \sq e_1: \tau_1 \rightarrow \tau_2 \\
\Gamma \vdash \E \tr exp_2 \sq e_2:\tau_1} {
\Gamma \vdash \E \tr exp_1\ exp_2  \sq e_1e_2 : \tau_2 } \]

\[ \inferrule[Ttuple]{
\Gamma \vdash \E \tr exp_1 \sq e_1 : \tau_1 \ldots
\Gamma \vdash \E \tr  exp_n \sq e_n : \tau_n} {
\Gamma \vdash \E \tr \langle exp_1,\ldots,exp_n \rangle \sq
\langle e_1,\ldots,e_n \rangle \colon \times [\tau_1,\ldots,\tau_n]}\]

\[
\inferrule[Tlet] {
\Gamma \vdash \E \tr decls \sq M : \sigma \\
\Gamma, \alpha : \sigma \vdash E @ (\alpha/m:\sigma) \tr exp \sq e : \tau \\
\Gamma \vdash \tau : \textrm{T}
} {
\Gamma \vdash \E \tr \textrm{let } decls \textrm{ in } exp \sq (\textrm{let } \alpha/m = M \textrm{ in } e : \tau) : \tau
}
\]


\section{elabDecl}

\textrm{Ddata} is handled with \texttt{ElaborateDatatype.elabDatatype}.
\[
\inferrule[Dval] {
\Gamma \vdash \E \tr exp \sq e : \tau \\
\Gamma, \afail : \textrm{T} \tr \E @ (a/m:\sandw{\tau}) @ (\afail/\mfail:\sandw{\textrm{tag}_\unit})
\tr pat \textrm{ on } \textrm{Snd}(m) \textrm{ with } \textrm{Snd}(\mfail) \sq M : \sigma
} {
\Gamma \vdash \E \tr \textrm{val } pat = exp \sq
(\textrm{let } \afail/\mfail = \sandw{\textrm{newtag } \langle \rangle} \textrm{ in }
\textrm{let } \alpha/m = \sandw{e} \textrm{ in } M) : \sigma
}
\]


\[
\inferrule[Dtype] {
\Gamma \vdash \E \tr  ty\rightsquigarrow \tau
} {
\Gamma \vdash \E \tr \textrm{type } id = ty \rightsquigarrow \Min{\textrm{CON } id} \headp{\tau} : ((\textrm{CON } id) : \headp{ S(\tau) })
}
\]

\[
\inferrule[Dopen] {
\Gamma \vdash \E \tr longid \textrm{ in MOD} \sq M : \sigma
} {
\Gamma \vdash \E \tr \textrm{open } id \sq M : \sigma
}
\]

\[
\inferrule[Dlocal] {
\Gamma \vdash \E \tr dec_1 \sq M_1 : \sigma_1 \\
\Gamma, \alpha : \sigma_1 \vdash \E @ (\alpha/(\textrm{out } m) : \sigma_1) \tr dec_2 \sq M_2 : \sigma_2
} {
\Gamma \vdash \E \tr \textrm{local } dec_1 \textrm{ in } dec_2 \textrm{ end} \sq
\\ \langle \alpha/m = \Min{\textrm{HIDE}} M_1, M_2 \rangle : \exists \alpha : (\textrm{HIDE} : \sigma_1) . \sigma_2
}
\]

\[
\inferrule[Dmodule] {
\Gamma \vdash \E \tr mod \sq M : \sigma
}{\Gamma \vdash \E \tr \textrm{structure } id = mod \sq \Min{\textrm{MOD } id} M : (\textrm{MOD } id : \sigma) }
\]

Dfun first requires some definitions:

\[ \varphi_\tau = \mu \alpha. \alpha \rightarrow ((\unit \rightarrow \tau) \rightarrow \tau) \rightarrow \tau \]
\[ \theta_\tau  = \textrm{roll}_{\varphi_\tau} \left(\lambda z : \varphi_\tau .\ \lambda f : (\unit \rightarrow \tau) \rightarrow \tau.\  f \left(\lambda \_ : \unit .\ ((\textrm{unroll } z) z) f  \right) \right) \]
\[ \Theta_\tau = (\textrm{unroll } \theta_\tau) \theta_\tau \]
\[ \textrm{fix}_\tau x.\ e =\left(\lambda \_ : \unit.\ \Theta_\tau (\lambda x' : \unit \rightarrow \tau. \ \textrm{let } x = x' \textrm{ in } e )    \right) \langle \rangle \]

\[\inferrule[Dfun]{
\Gamma \vdash \E \tr  ty_1\rightsquigarrow \tau_1 \\
\Gamma \vdash \E \tr  ty_2\rightsquigarrow \tau_2 \\
\Gamma, a : \left(  \Sigma \_: (\textrm{VAL }id_1 : \sandw{\tau_1 \rightarrow \tau_2}) . (\textrm{VAL } id_2 : \tau_1) \right) \vdash \E @ (a/m: \textrm{same Ssigma as added to $\Gamma$}) \tr \textrm{exp} : \tau_1
}
{
\Gamma \vdash \E \tr \textrm{fun} id_1(id_2 : ty_1) : ty_2 = \textrm{exp} \sq  \\
\Min{\textrm{VAL } id_1}  \sandw{\textrm{fix}_{\tau_1  \rightarrow \tau_2} f. \left( 
\lambda x : \tau_1.
\textrm{let } \alpha/m =  \langle \_/\_ = \Min{\textrm{VAL }id_1} \sandw{f \langle \rangle}, \Min{\textrm{VAL }id_2} \sandw{x} \rangle \textrm{ in } e : \tau_2
\right) }
\\ : (\textrm{VAL } id_1 : \sandw{\tau_1\rightarrow\tau_2})
}
\]




\section{elabDecls}
\[
\inferrule[nil]{ }
{\Gamma \vdash \E \tr \epsilon \sq * : 1 }
\]
\[
\inferrule[decl :: decls]{
\Gamma \vdash \E \tr dec_1 \sq M_1 : \sigma_1
\\ \Gamma, \alpha : \sigma1 \vdash \E @ (\alpha/m:\sigma_1) \tr dec_2 \sq M_2 : \sigma_2}
{ \Gamma \vdash \E \tr dec_1 \ dec_2 \sq \langle \alpha/m=M_1, M_2 \rangle : \Sigma \alpha : \sigma_1 . \sigma_2 }
\]

\section{elabModule}

The code for \texttt{Mseal} is written for us. You get a module, opacity, and signature; call \texttt{elabModule} and \texttt{elabSg}, and then pass the
results (and the opacity) to \texttt{Ascribe.ascribe}.

\[
\inferrule[Mident]{
\Gamma \vdash \E \tr longid \textrm{ in MOD} \sq M : \sigma}{
\Gamma \vdash \E \tr longid \sq M : \sigma}
\]
\[
\inferrule[Mstruct]{
\Gamma \vdash \E \tr dec \sq M : \sigma}{
\Gamma \vdash \E \tr \textrm{struct } dec \textrm{ end} \sq M : \sigma}
\]

\end{document}